\chapter{Fundamentals and Related Work\label{cha:chapter2}}

%%%%%%%%%%%%%%%
Electronic data processing is applied math and relies therefor on data of a certain, deterministic structure. In mathematics and computer science such rules of data representation are called canonical or normal form. Canonicalization is the discipline of transforming data of possible different forms of representation into a standardized form. An illustrative example of canonicalization is the representation of a boolean variable described in the XML schema type definition. By definition a boolean variable supports binary logic representation, meaning that some state or flag can either be true or false. The set of possible literals might contain: 1, 0, true and false. The XML schema type definition defines the canonical form of boolean as true and false, whereas 1 can be mapped to true and 0 to false. Data in a normalized form can be transfered in any other form.

The foundation of the idea of storing large data in a reusable format were captured by Edgar F. Codd in 1970 with his proposal for database normalization. The concept behind the proposal is to break the information down into entities with attributes and describe relations among them. A second and third version of database normalization followed. The entities person and address can be considered as example. The person entity has the attributes surname and name, the address entity the attributes street, zip code and city. In addition the 
%%%%%%%%%%%%%%%%%%%%


This section is intended to give an introduction about relevant terms, technologies and standards in the field of X. You do not have to explain common technologies such as HTML or XML. 

\section{Technologies \label{sec:tech}}

This section describes relevant technologies, starting with X followed by Y, concluding with Z.

\subsection{Technology A\label{sec:aaa}}

It's always a good idea to explain a technology or a system with a citation of a prominent source, such as a widely accepted technical book or a famous person or organization. 

Exmple: Tim-Berners-Lee describes the ''WorldWideWeb'' as follows:
\\
\textit{''The WorldWideWeb (W3) is a wide-area hypermedia information retrieval initiative aiming to give universal access to a large universe of documents.''} \cite{timwww}
\\
\\
You can also cite different claims about the same term.
\\
According to Bill Gates \textit{''Windows 7 is the best operating system that has ever been released''} \cite{billgates} (no real quote)
In opposite Steve Jobs claims Leopard to be \textit{''the one and only operating system''} \cite{stevejobs}

If the topic you are talking about can be grouped into different categories you can start with a classification.
Example: According to Tim Berners-Lee XYZ can be classified into three different groups, depending on foobar \cite{timwww}:
	\begin{itemize}
		\item Mobile X
				\vspace{-0.1in} 
		\item Fixed X
				\vspace{-0.1in} 
		\item Combined X
 	\end{itemize}

\subsection{Technology B\label{sec:bbb}}

For internal references use the 'ref' tag of LaTeX. Technology B is similar to Technology A as described in section \ref{sec:aaa}.

\newpage

\subsection{Comparison of Technologies\label{sec:comp}}

\begin{table}[htb]
\centering
\begin{tabular}[t]{|l|l|l|l|}
\hline
Name & Vendor & Release Year & Platform \\
\hline
\hline
A & Microsoft & 2000 & Windows \\
\hline
B & Yahoo! & 2003 & Windows, Mac OS \\
\hline
C & Apple & 2005 & Mac OS \\
\hline
D & Google & 2005 & Windows, Linux, Mac OS \\
\hline
\end{tabular}
\caption{Comparison of technologies}
\label{tab:enghistory}
\end{table}

\section{Standardization \label{sec:standard}}

This sections outlines standardization approaches regarding X.

\subsection{Internet Engineering Task Force\label{sec:itu}}

The IETF defines SIP as '...' \cite{rfcsip}

\subsection{International Telecommunication Union\label{sec:itu}}

Lorem Ipsum...

\subsection{3GPP\label{sec:3gpp}}

Lorem Ipsum...

\subsection{Open Mobile Alliance\label{sec:oma}}

Lorem Ipsum...

\section{Concurrent Approaches \label{sec:summ}}

There are lots of people who tried to implement Component X. The most relevant are ...