\thispagestyle{empty}
\vspace*{1.0cm}

\begin{center}
    \textbf{Abstract}
\end{center}

\vspace*{0.5cm}

\noindent
The thesis is about to introduce a framework that automates the feature extraction of semi-structured information entities. In a world of ever growing amount of data digitally available the diversity of data is reaching unprecedented levels. For appropriate and most beneficial usage of data that are aggregated across several sources it is essential to find a unified representation to satisfy the downstream applications demand for data purity. Instances exist that highly depend on the quality of input data but also require a broad mass of information. The initial focus lies on the structuring of the problems within distributed and messy data. A breakdown of the wide spectrum of issues results in a a concrete and generalized categorization of data problems.  This categorization fits modern technologies of data storage as well as well known and aged approaches and provides a solid bases for the further work. Irrespective of underlying systems, data formats or structures, a the scope and target is defined. 

In a problem driven approach the initial problem categorization is used to define the requirements for novel framework handling heterogeneous data with the aim of transforming them into valuable representations. The requirements are related to technical as well as economical aspects and set the fundamentals for a intuitive concept. In focus is the automation and the extendability of the final solution. With the outlook for a reference implementation a pragmatic concept is defined. Here modularized approach is chosen to generate the most possible coverage of the beforehand stated demands. The requirement engineering as well as the concept finding has a strong focus of the problem environment and is detached from technical details. Through an reference implementation the concept is challenged with respect to the defined requirements. Questions regarding technical foundations, libraries and frameworks are addressed. A clean technical concept is found and a intuitive abstraction of the underlying problem can is set up. 

A comprehensive evaluation points out the applicability and realizability of the proposed framework ans shows the success of the problem driven approach. Furthermore majority of the evaluation is devoted to the assessment of the systems learning ability. Under this point a toy dataset is introduced that measures the performance while it proves the framework for its readiness to use. The final conclusion summarized this work and assesses the use of the framework through an overall viewpoint. After presenting encounter problems a lookout for further work is given.