\thispagestyle{empty}
\vspace*{1.0cm}

\begin{center}
    \textbf{Abstract}
\end{center}

\vspace*{0.5cm}

\noindent
The thesis is about to introduce a framework that automates the feature extraction from semi-structured information entities. In a world of an ever growing amount of data digitally available their diversity is reaching unprecedented levels. Instances exist that highly depend on the quality of input data but also require a broad mass of information. For appropriate and most beneficial usage of data that are aggregated across several sources it is essential to find a unified representation to satisfy the demand for data purity of an application. The initial focus of this work lies on the structuring of the problems of distributed and messy data. A breakdown of the wide spectrum of issues results in a a concrete and generalized categorization of data problems.  This categorization proofs to be aligned with modern data storage technologies as well as well-established and aged storage approaches. The analysis provides a solid basis for the rest of the work. Irrespective of underlying systems, data formats or structures, the scope and target is defined. 

In a problem driven approach the initial data issue categorization is used to define the requirements for novel framework handling heterogeneous data with the aim of transforming them into valuable representations. The requirements are related to technical as well as economical aspects and set the fundamentals for an intuitive concept. In the focus is the automation and the extensibility of the final solution. With the outlook for a reference implementation a pragmatic concept is defined. Here a modularized approach is chosen to generate the most possible coverage of the beforehand stated demands. The requirement engineering as well as the concept finding has a strong focus of the problem environment and is detached from technical details. Through a reference implementation the concept is challenged with respect to the elaborated requirements. Questions regarding technical foundations, libraries and frameworks are addressed. A clean technical concept is found and an intuitive abstraction of the underlying problem is set up. 

Finally, a comprehensive evaluation points out the applicability and feasibility of the proposed framework and shows the success of the problem driven approach. Furthermore, the majority of the evaluation is devoted to the assessment of the systems learning ability. Under this point a toy dataset is introduced that measures the performance while it proves the frameworks readiness to use. The conclusion summarizes this work and assesses the use of the framework through an overall viewpoint. After presenting encounter problems a lookout for further work is given.